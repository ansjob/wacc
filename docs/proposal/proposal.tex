\documentclass[a4paper,11pt]{article}
\usepackage[english]{babel}
\usepackage[T1]{fontenc}
\usepackage{fancyhdr}
\usepackage{graphicx}
\usepackage{a4wide}
\usepackage{numprint}
\usepackage{verbatim}
\usepackage{moreverb}
\usepackage{url}
\usepackage{cite}
\pagestyle{fancy}

\fancyhead[L]{
	\textbf{ID2213} : Logic Programming : Proposal for project \\
	Marcus Larsson (\url{marcular@kth.se}) and
	Andreas Sj�berg (\url{ansjob@kth.se})
	}

\fancyhead[R]{}
\newcommand{\tab}{\hspace*{2em}}

\begin{document}

\section{Background}

We both took the course on software semantics last semester at Valhallav�gen,
where the programming language \verb|while| was considered.
The course book used defines a compilation function to an abstract machine language which then can be executed using a runtime model also defined in the book.
What we would like to do as an exercise in Prolog is to build a lexer, parser, compiler, and runtime for \verb|while|.
We did consider doing some proofs on while-programs, but as that was part of the semantics course, we would like to do something different,
like implementing a non-trivial lexer/parser in Prolog.
If time permits, we may extend the runtime to symbolic execution, and do some proofs, but this would be the lowest priority in this project.

\end{document}
