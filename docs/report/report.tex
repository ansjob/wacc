\documentclass[a4paper,11pt]{article}
\usepackage[english]{babel}
\usepackage[T1]{fontenc}
\usepackage{fancyhdr}
\usepackage{graphicx}
\usepackage{a4wide}
\usepackage{numprint}
\usepackage{verbatim}
\usepackage{moreverb}
\usepackage{url}
\usepackage{cite}
\usepackage{listings}
\pagestyle{fancy}
\bibliographystyle{plain}

\fancyhead[L]{
	\textbf{ID2213} : Logic Programming : Project report \\
	Marcus Larsson (\url{marcular@kth.se}) and
	Andreas Sj�berg (\url{ansjob@kth.se})
	}

\fancyhead[R]{}
\newcommand{\tab}{\hspace*{2em}}

\begin{document}

\section{Background}

We both took the course on software semantics last semester at Valhallav�gen,
where the programming language \verb|while| was considered.
The course book \cite{semantics} used defines a compilation function to an abstract machine language which then can be executed using a runtime model also defined in the book.

\section{Prerequisites}

\subsection{The while language}

The while language is a small imperative toy-language described in \cite{semantics}.
The core version of the language has the concepts of assignments, conditionals, loops, and a skip.
As an assignment in the semantics course, we extended this with division and try-catch statements, with the obvious exception being divison by zero.

The language is executed in an Abstract Machine (AM), which is a stack-based machine with associative memory. This machine works on instructions that are not part of the language syntax, so a compilation step is necessary. The instruction set, as well as an abstract description of the compilator is available in \cite{semantics}.


\subsection{Symbolic Execution}

As described in \cite{semantics} we can analyze certain aspects of a program using symbolic execution. The configuration graph (all possible configurations/executions) are explored and the least upper bound is computed for the memory at each statement of a program with regards to some domain.

During this project we implemented two domains; the detection of signs domain, and parity domain for detecting odd/even numbers.

\section{Implementation}

The project can be divided into some relatively independent modules, each described in their own subsection below.

\subsection{Lexer}
The lexer simply converts the plain-text source file into a list of relevant tokens.
We implemented this using Prolog's built in DCG feature, simply defining the keywords and tokens for the language.

\subsection{Parser}
The parser converts a list of tokens into an abstract syntax tree (AST), also known as a parse tree.
This is also implemented using as a DCG, and describes what a legal statement in the language is.
The language is described on EBNF in \cite{semantics}, and writing the parser was simply a matter of converting this into DCG.

\subsection{Compiler}
The compiler converts statements into instructions.
We implemented this as a simple recursive prolog predicate, that takes an AST as input and produces a list of instructions as output.
The compilation function is defined in \cite{semantics}.

\subsection{Runtime}
The runtime consists of a single predicate \verb|step/2| that takes an initial configuration and gives the next one.
This function is also defined in the course book, and implementing it was simply a matter of translation.
A full program execution consists of repeating the \verb|step| predicate until no code remains to be executed. This is done by the predicate \verb|run_code/3|. After a program has been run, we simply print the memory of the final configuration, which is the output of the program.

\subsection{Analyzer}
The semantics course had a focus on automatically analyzing programs, specifically using symbolic execution.
We implemented this by extending the \verb|step| predicate to accept what domain to work in, and made it non-deterministic.
This means we can collect all next steps using Prolog's \verb|findall| predicate.
To avoid looping indefinitely, we keep track of the set of visited configurations.
This is implemented using the ordered set predicates built into SWI-Prolog.

One thing needed to compute the least upper bound of the memory at each statement is the concept of control points.
We simply assign an integer to each instruction that corresponds to a statement.
For example the following would be a control point assignment:
\begin{verbatim}
1: while (true) do
2:   if( x <= 5) then
3:      skip
     else
4:      skip
\end{verbatim}

We created a predicate \verb|add_control_points| that adds control points to an AST.
Then at the compilation step, we accept such a ''decorated'' AST, and add the control points to the corresponding AM instructions.

\section{Issues resolved}


\subsection{Exploring the configuration graph in an elegant manner}
One of the main things we like about the implementation is the use of Prolog's ability 
to find several solutions to a predicate in the analysis. 
For example, in the detection of signs analysis, a positive number subtracted from a positive number could result in any sign of the difference.
This is simply stated by the following predicate:
\begin{verbatim}
sub(sign, pos, pos, X):- member(X, [neg, zero, pos]).
\end{verbatim}

And then one of the clauses in the \verb|step| predicate is:
\begin{verbatim}
step(Domain, config([sub|Code], [A, B|Stack], Store), config(Code, [Diff|Stack], Store)):-
    sub(Domain, A, B, Diff).
\end{verbatim}

Note that \verb|step| then has the same number of solutions as the \verb|sub/3| predicate.
This means that we could implement the regular execution as a special case where each of these operations only had one solution (we call this the \verb|int| domain).

\subsection{Hot-switching domains}

We had issues with compilation warnings when compiling the different domain operations, since they all shared predicate name and arity.
The solution involved some nice meta-programming using \verb|functor/3| and \verb|arg/3|, that allows us to have different functors for each domain,
avoiding compilation issues.

For example \verb|sub(sign, X, Y, Z)| will make a call to \verb|sign_sub(X, Y, Z)|.
This also allows for easy extension with new domains, as a proof of this, we added the parity domain (also known as odd/even) in an afternoon.

\subsection{Fixing ambiguity in the program grammar}
The grammar presented in the book is ambiguous with regards to being left or right associative.
For example consider the statement \verb|while true do skip; x := 5|. Is the assignment part of the loop body or not?
It can be interpreted as
\begin{verbatim}
while (true) do
    skip;
x:= 5
\end{verbatim}
or

\begin{verbatim}
while (true) do
    skip;
    x:= 5
\end{verbatim}

We simply chose to make the language left-associative, making the first variant the one that is executed.
This is because we wanted to be consistent with for example C and Java.

\section{Evaluation}

This project has been fun to work on, and we have worked on many aspects of the prolog language.
We have used meta-programming, DCGs, tree-traversal, signal handling, and output formatting, among other things.
This means we have gotten to see most of the strengths of prolog, and some of it's weaknesses.
One thing we have not used as much as we would have liked to is the capability to mix input and output parameters in the same predicate.

\clearpage
\bibliography{report}

\clearpage

\appendix
\section{Source Code}
\lstset{numbers=left, breaklines=true}

\lstinputlisting[caption=analyzer.pl, language=Prolog]{../../analyzer.pl}
\clearpage
\lstinputlisting[caption=atomics.pl, language=Prolog]{../../atomics.pl}
\clearpage
\lstinputlisting[caption=autotest.pl, language=Prolog]{../../autotest.pl}
\clearpage
\lstinputlisting[caption=code\_printer.pl, language=Prolog]{../../code_printer.pl}
\clearpage
\lstinputlisting[caption=compiler.pl, language=Prolog]{../../compiler.pl}
\clearpage
\lstinputlisting[caption=domain\_int.pl, language=Prolog]{../../domain_int.pl}
\clearpage
\lstinputlisting[caption=domain\_oddeven.pl, language=Prolog]{../../domain_oddeven.pl}
\clearpage
\lstinputlisting[caption=domain\_sign.pl, language=Prolog]{../../domain_sign.pl}
\clearpage
\lstinputlisting[caption=domains.pl, language=Prolog]{../../domains.pl}
\clearpage
\lstinputlisting[caption=grammar.pl, language=Prolog]{../../grammar.pl}
\clearpage
\lstinputlisting[caption=lexer.pl, language=Prolog]{../../lexer.pl}
\clearpage
\lstinputlisting[caption=main.pl, language=Prolog]{../../main.pl}
\clearpage
\lstinputlisting[caption=maps\_to\_strings.pl, language=Prolog]{../../maps_to_strings.pl}
\clearpage
\lstinputlisting[caption=runtime.pl, language=Prolog]{../../runtime.pl}
\clearpage
\lstinputlisting[caption=tester.pl, language=Prolog]{../../tester.pl}
\clearpage
\lstinputlisting[caption=variable\_extractor.pl, language=Prolog]{../../variable_extractor.pl}
\clearpage
\lstinputlisting[caption=wanalyze.pl, language=Prolog]{../../wanalyze.pl}
\clearpage


\end{document}
