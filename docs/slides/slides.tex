
\documentclass[]{beamer}

\usepackage{beamerthemesplit} 
\usepackage[utf8]{inputenc}
\usepackage{algpseudocode}
\usepackage{algorithm}
\usepackage{moreverb}
\usepackage{listings}

\title{WACC - The While Analysis and Compilation Chain}
\author{Andreas Sjöberg \and Marcus Larsson}
\institute{KTH}
\date{\today}

\begin{document}

\begin{frame}
  \titlepage
\end{frame}

\section[Outline]{}

\begin{frame}
  \tableofcontents
\end{frame}

\section{Prerequisites}

\subsection{The while language}
\begin{frame}[fragile]
	\frametitle{While - a small programming language}
	\begin{itemize}
		\item C-like syntax
			\begin{itemize} 
				\item assignments
				\item conditionals -  \verb|if then else |
				\item loops - \verb|while do|
				\item try-catch blocks
				\item \verb!skip!
			\end{itemize}
		\item Only integer variables ($Var \subset \mathbb{Z} $)
	\end{itemize}
\end{frame}

\begin{frame}[fragile]
	\frametitle{A (very) small example}
\begin{boxedverbatim}
y := x * x
\end{boxedverbatim}
\end{frame}

\section{Compiling and running}

\subsection{Lexing}
\begin{frame}[fragile]
\frametitle{Lexing}
\begin{itemize}
	\item Turn stream of bytes into list of tokens.
	\item Implemented using DCG.
	\item Our small example:
\begin{boxedverbatim}
?- lex(Tokens, "y := x * x", []).
Tokens = [var(y), =, var(x), *, var(x)]
\end{boxedverbatim}
\end{itemize}
\end{frame}

\subsection{Parsing}
\begin{frame}[fragile]
\frametitle{Parsing}
\begin{itemize}
	\item Turn list of tokens into an Abstract Syntax Tree (AST)
	\item Also done using DCG
	\item Our small example again
\begin{boxedverbatim}
?- lex(Tokens, "y := x * x", []),
   stmt(AST, Tokens, []).
Tokens = [var(y), =, var(x), *, var(x)],
AST = ass(var(y), var(x)*var(x))
\end{boxedverbatim}
\end{itemize}
\end{frame}

\subsection{Compiling}

\begin{frame}[fragile]
\frametitle{On Abstract Machines}
\begin{itemize}
	\item Compile the AST into a list of Abstract Machine (AM) instructions
	\item More general (and easier) than actual assembly instructions
	\item Stack-based machine (unlimited)
	\item Associative memory (unlimited)
	\item Could with relative ease be converted to actual machine code
\end{itemize}
\end{frame}

\begin{frame}[fragile]
\frametitle{Compiling}
\begin{itemize}
	\item Implemented as a binary predicate \verb|compile/2|
	\item Our example (continued)
\begin{boxedverbatim}
?- lex(Tokens, "y := x * x", []), 
   stmt(AST, Tokens, []), 
   compile(AST, Code).
Tokens = [var(y), =, var(x), *, var(x)],
AST = ass(var(y), var(x)*var(x)),
Code = [fetch(x), fetch(x), mult, store(y)]
\end{boxedverbatim}
\end{itemize}
\end{frame}

\subsection{Running}
\begin{frame}[fragile]
\frametitle{Configurations}

\begin{itemize}
	\item A configuration is a state of the AM
	\item $Config = Code \times Stack \times Store$
	\item Example: $\langle [add, store(x)], [5, 3], []  \rangle$
\end{itemize}
\end{frame}

\begin{frame}[fragile]
\frametitle{Stepwise execution}
\begin{itemize}
	\item Execution in the AM is simply a series of configurations
	\item AM finds the next configuration $C_1$ from a given configuration $C_0$ with the predicate \verb|step/2|
	\item Written as a set of transition rules
\begin{boxedverbatim}
step(
	  config([add|Code], [A,B|Stack], Store), 
	  config(Code, [Sum|Stack], Store)
	):-
	Sum is A + B.
\end{boxedverbatim}
\end{itemize}
\end{frame}

\begin{frame}

\frametitle{Live demo}
\centering
Live demonstration
\end{frame}

\section{Analyzing}

\begin{frame}[fragile]
\frametitle{Going further}
\centering
Okay, that was fun. Let's do something interesting.
\end{frame}

\subsection{Program Analysis}

\begin{frame}
\frametitle{Program Analysis}
\begin{itemize}
	\item By analyzing the program, what can we say about it?
	\begin{itemize}
		\item Termination
		\item Possible variable assignments
		\item Exceptions
		\item Dead code
	\end{itemize}
\end{itemize}
\end{frame}

\subsection{Domains}

\begin{frame}
\frametitle{Domains}
\begin{itemize}
\item We want to explore the configuration graph of a program, but this is typically infinite
\item Restrict ourselves to some finite domain, e.g.
	\begin{itemize}
	\item Sign (negative, zero, positive)
	\item Parity (odd, even)
	\end{itemize}
\item The configuration graph is now finite!
\end{itemize}
\end{frame}

\subsection{Symbolic Execution}

\begin{frame}[fragile]
	\frametitle{Symbolic Execution}
	\begin{itemize}
		\item We must traverse the configuration graph
		\item Idea: use the \verb|step| predicate to expand nodes
		\item Modify \verb|step| to take the current domain into consideration
		\item \verb|step| may then have several solutions, explore them all using \verb|findall|
		\item Keep track of already visited configurations to avoid loops
		\item The interesting configurations are before and after statements (control points) 
	\end{itemize}
\end{frame}

\subsection{Least Upper Bound (lub)}
\begin{frame}[fragile]
	\frametitle{Learning from our explorations}
	\begin{itemize}
		\item Look at memory at all control points, what can we say about the variables?
		\item If a control point is reached with \verb|x = neg| and \verb|x = pos|, we can say \verb|x = non_zero|
		\item Find the most specific, but still safe thing to say (the Least Upper Bound)
	\end{itemize}
\end{frame}

\begin{frame}[fragile]
	\frametitle{Learning from our explorations}
	\centering
	Live demonstration
\end{frame}


\end{document}
